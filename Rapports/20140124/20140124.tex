\documentclass[12pt,a4paper]{article}
\usepackage{mathced}
\usepackage{graphicx}
\usepackage{eurosym}
\usepackage[top=2cm, bottom=2cm, left=2cm, right=2cm]{geometry}
\usepackage{pgfplots}
\author{HUGUENIN Cédric}
\everymath{\displaystyle}
\begin{document}
\entete{TELECOM Nancy}{PIDR}{Premier contact}{24 janvier 2014}

\section{Données à conserver}

\begin{itemize}
\item \'Enoncé de l'exercice ;
\item code de l'élève ;
\item nombre tentatives ;
\item temps entre chaque tentatives ;
\item changement d'exercice alors que l'actuel n'est pas résolu.
\end{itemize}

\section{Structure stockage}

\begin{itemize}
\item Un dépôt par utilisateur ;
\item une branche par leçon ;
\item une branche par exercice dans chaque branche de leçon.
\end{itemize}
$ $\\
\begin{itemize}
\item Message du commit : FAIL ou SUCCESS, l'ID de l'exercice, date, langage, indice utilisé.
\item Contenu du commit : code source, le message d'erreur s'il existe et l'énoncé.
\end{itemize}

Dans le code source, il faut sauvegarder le template.

Commit lors de la compilation.
Quand il est en ligne, il push.

\section{Questions}

\begin{itemize}
\item Serveurs http://www.loria.fr/\~oster/plm-cloud/, et http://jlmserver-chmod0.appspot.com ?
\item Utilisation de Play ? Utilité de Google App Engine (a priori non compatible avec Play 2.x) ?
\end{itemize}

\end{document}