\documentclass[12pt,a4paper]{article}
\usepackage{mathced}
\usepackage{graphicx}
\usepackage{eurosym}
\usepackage[top=2cm, bottom=2cm, left=2cm, right=2cm]{geometry}
\usepackage{pgfplots}
\author{HUGUENIN Cédric}
\everymath{\displaystyle}
\begin{document}
\entete{TELECOM Nancy}{PIDR}{Serveur d'application PLM}{}
% Section : les dates
% Titre du PIDR en haut
\section{15 janvier 2014}
\subsection{Données à conserver}

\begin{itemize}
\item \'Enoncé de l'exercice ;
\item code de l'élève ;
\item nombre tentatives ;
\item temps entre chaque tentatives ;
\item changement d'exercice alors que l'actuel n'est pas résolu.
\end{itemize}

\subsection{Structure stockage}

\begin{itemize}
\item Un dépôt par utilisateur ;
\item une branche par leçon ;
\item une branche par exercice dans chaque branche de leçon.
\end{itemize}
$ $\\
\begin{itemize}
\item Message du commit : FAIL ou SUCCESS, l'ID de l'exercice, date, langage, indice utilisé, nombre de tests passés.
\item Contenu du commit : code source, le message d'erreur s'il existe et l'énoncé.
\end{itemize}

Fichiers avec le code source, le template et le résultat.

Commit lors de la compilation.
Quand il est en ligne, il push.

\subsection{Questions}

\begin{itemize}
\item Serveurs \url{http://www.loria.fr/~oster/plm-cloud/}, et \url{http://jlmserver-chmod0.appspot.com} ?
\item Utilisation de Play ? Utilité de Google App Engine (a priori non compatible avec Play 2.x) ?
\end{itemize}

\section{24 janvier 2014}

Plutôt une branche par exercice.
Google App Engine : première tentative.
Play plus simple et plu portable.
Ne pas dépendre d'une technique.
 
Voir ce que l'on attend de nous

Notre rapport doit permettre de faire gagner du temps à la prochaine équipe

Gestion de projet : ToDo list, planning des grandes étapes.

Regarder le code de la session de la PLM
Git pourrait faire l'espion et la session

Mode professeur à reprendre. Fonctionnalités attendues (serveur d'utilisation) :
\begin{itemize}
\item explorer graphiquement l'évolution de chaque élèves ;
\item mise en forme des données : représentation ;
\item 
\end{itemize}
Première version pour monter les gits. (premier tiers de notre travail)
Commit en local.
Monter automatiquement sur le serveur.
Moulinette d'extraction des données.
% modélisation apprenante. Alertes configurables (nombre essai, temps passé)
% openbadges.org
% Besoins d'authentification des élèves.
% 
% Détection d'exercices infaisable : personne n'y arrive

% Récupérer des leçons sur un serveur

% Trois cibles : apprenants, auteurs de ressources : création de contenu, enseignants : faire leur séquences d'exo : éditeur d'exercice et de séquences.

% Traces utilisées en live : savoir qui aider, et après coup, pour repérer les exercices qui marchent ou pas.

% Prendre le controle à distance de la machine de l'élèves / chat vidéo

% nouveaux langages : scratch, js, C

% forum collaboratif : commenter et proposer des exercices (les élèves par exemple). On propose aux autres élèves et les notes. (Recapcha) 

% Serveur de centralisation

% ToDo list :
Faire le git en local et le push sur un serveur

Trouver des étapes : établir un planning.

Faire un serveur qui centralise le code des élèves.

Gestionnaire de sessions en GIT qui commit plein de trucs
Template, source, énoncé dans un seul commit.
Chargeur qui récupère les données depuis le git.

ImportCloudSession.java
SessionDB.java Exo, language, fichier : en mémoire ; le ISessionKit : gère l'écriture sur le disque.
Exercice.java -> SourceFile.java : template et le body (ce que l'élève à tapé) actuel
Finir de mettre dans SessionDB le code de l'élève
SourceFile : ce qui vient du jar
SessionDB : ce qu'il faut écrire sur le disque
Qui est en charge de getCompilableContent ?


Code session.
Ramener code de l'élève dans SessionDB
Les tests doivent passer
Contenu du sessionDB dans le git et le contenu du template.

Bonus :
Poster sur GitHub pour le feedback des problèmes : dans les issues
PLM doit se loguer sur GitHub pour poster les messages d'erreurs.

Attentes du stages
\end{document}