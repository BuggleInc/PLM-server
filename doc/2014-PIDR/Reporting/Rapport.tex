\documentclass[12pt,a4paper]{article}
\usepackage{mathced}
\usepackage{graphicx}
\usepackage{eurosym}
\usepackage[top=2cm, bottom=2cm, left=2cm, right=2cm]{geometry}
\usepackage{pgfplots}
\author{HUGUENIN Cédric}
\everymath{\displaystyle}
\begin{document}
\entete{TELECOM Nancy}{PIDR}{Serveur d'application PLM}{}
% Section : les dates
% Titre du PIDR en haut
\section{15 janvier 2014}
\subsection{Données à conserver}

\begin{itemize}
\item \'Enoncé de l'exercice ;
\item code de l'élève ;
\item nombre tentatives ;
\item temps entre chaque tentatives ;
\item changement d'exercice alors que l'actuel n'est pas résolu.
\end{itemize}

\subsection{Structure stockage}

\begin{itemize}
\item Un dépôt par utilisateur ;
\item une branche par leçon ;
\item une branche par exercice dans chaque branche de leçon.
\end{itemize}
$ $\\
\begin{itemize}
\item Message du commit : FAIL ou SUCCESS, l'ID de l'exercice, date, langage, indice utilisé, nombre de tests passés.
\item Contenu du commit : code source, le message d'erreur s'il existe, le template et l'énoncé.
\end{itemize}

Fichiers avec le code source, le template et le résultat.

Commit lors de la compilation.
Quand il est en ligne, il push.

\subsection{Questions}

\begin{itemize}
\item Serveurs ?
	\begin{itemize}
	\item \url{http://www.loria.fr/~oster/plm-cloud/}
	\item \url{http://jlmserver-chmod0.appspot.com}
	\end{itemize}
\item Utilisation de Play ? Utilité de Google App Engine (a priori non compatible avec Play 2.x) ?
\end{itemize}

\section{24 janvier 2014}
\subsection{Git}
Git pourrait faire l'espion et la session.

Il faut plutôt faire une branche par exercice.

%Gestionnaire de sessions en GIT qui commit plein de trucs.
Template, source, énoncé dans un seul commit.
Chargeur qui récupère les données depuis le git.
Faire le git en local et le push sur un serveur. 

Première version pour monter les gits : (premier tiers de notre travail). Commit en local. Monter automatiquement sur le serveur.
\subsection{Modification du code de la PLM}
Regarder le code de la session de la PLM.

Besoins d'authentification des élèves.

ImportCloudSession.java
SessionDB.java Exo, language, fichier : en mémoire ; ISessionKit gère l'écriture sur le disque.
Exercice.java -> SourceFile.java : template et le body (ce que l'élève à tapé) actuel.
Finir de mettre dans SessionDB le code de l'élève.
SourceFile : ce qui vient du jar.
SessionDB : ce qu'il faut écrire sur le disque.
Qui est en charge de getCompilableContent ?

Ramener code de l'élève dans SessionDB.

Les tests doivent passer.

\subsection{Mode professeur}
Mode professeur à reprendre. Fonctionnalités attendues (serveur d'utilisation) :
\begin{itemize}
\item explorer graphiquement l'évolution de chaque élèves ;
\item mise en forme des données : représentation ;
\item alertes configurables (nombre essai, temps passé) ;
\item détection d'exercices infaisable : personne n'y arrive ;
\item traces utilisées en direct : savoir qui aider, et après coup, pour repérer les exercices qui marchent ou pas.
\end{itemize}

\subsection{Autres}

Google App Engine était une première tentative. Play plus simple et plus portable. Il faut éviter de dépendre d'une technique.

Système de badges : \url{openbadges.org}

Récupérer des leçons sur un serveur.

Notre rapport doit permettre de faire gagner du temps à la prochaine équipe. Rapport à rendre pour le 22 mai 2014. Soutenance : 28 mai 2014

Gestion de projet : ToDo liste, planning des grandes étapes.

Trois cibles : apprenants, auteurs de ressources : création de contenu, enseignants : faire leur séquences d'exo : éditeur d'exercice et de séquences.
Moulinette d'extraction des données. Modélisation apprenante.

Prendre le contrôle à distance de la machine de l'élèves / chat vidéo

Nouveaux langages : scratch, js, C.

Forum collaboratif : commenter et proposer des exercices (les élèves par exemple). On propose aux autres élèves et les notes. (Recapcha) 


Bonus :
Poster sur GitHub pour le feedback des problèmes : dans les issues.
PLM doit se loguer sur GitHub pour poster les messages d'erreurs.
\url{http://developer.github.com/v3/issues/#create-an-issue}

\subsection{Questions}
\begin{itemize}
\item Différences entre SessionDB et ExecutionProgress ?
\item Utilisation exclusive du git pour stocker le code en local ?
\item Le serveur alerte le professeur d'un élève en difficultés à partir des données du git pushé en temps réel ? (Commit log en JSON -> données récupérables avec un fetch moins coûteux qu'un pull)
\end{itemize}

\section{Mercredi 26 mars 2014}

%Une branche par élèves mais attention, plusieurs appareils par élèves.
%
%Seul le prof peut retrouver ses élèves.

Le code est un sur un serveur git public :
\begin{itemize}
\item il n'y a aucune information personnelle ou identifiante ;
\item il est librement accessible en lecture/écriture car git assure que rien n'est détruit (taille max des commits pour décourager les attaquants et éviter les pb de quota disque) ;
\item les branches des élèves sont identifiées par le uuid stocké localement dans PLM.
\end{itemize}

Scénario :
\begin{enumerate}
\item les élèves peuvent pousser sur leur branche ;
\item un élève a plusieurs papareils ;
\item l'élève veut connecter/rattacher plusieurs UUID ;
\item le prof veut suivre ses élèves ;
\item l'élève peut retrouver sa session sur une machine fraiche ;
\item au démarrage : 
	\begin{itemize}
	\item nouvelle partie anonyme ;
	\item continuer partie anonyme ;
	\item se connecter à une identité ;
	\item créer une nouvelle identité ;
	\end{itemize}
\item créer un UUID au premier démarrage ;
\item une fois identifié, revendiquer une partie anonyme.
\end{enumerate}

Il y a un mapping UUID(s) <-> identité optionnel et on peut le faire à postériori.

Scénario bonus :
\begin{itemize}
\item les élèves peuvent se créer une identité ;
\item l'élève veut voir la version de son code d'il y a deux jours.
\end{itemize}

%Java uuid est un identifiant unique de session stocké localement et qui est le nom de branche sur le serveur. Si plusieurs machines, on peut merge.

Il faut changer les ZipSessions et GitSessions :
\begin{itemize}
\item aucune écriture ;
\item lecture sur le même layout que ZipSession ;
\item attention : je change d'exo après avoir édité mais pas exécuté ;
\item attention : savoir rapidement quels exos réussi en quel langage.
\end{itemize}

Tables :
\begin{itemize}
\item une table des identités : clé, nom complet, UUID(s), mail, ... ;
\item une table authentifications : login, mot de passe, clé utilisateur ;
\item une table participants : id individu, id classe ;
\item une table classe : nom, descriptif, id classe, prof(s).
\end{itemize}



%
%Tout le monde peut tirer/pousser n'importe quoi. Ne pas autoriser les commits de plus de 4 ko par exemple. Git ne permet que d'ajouter de l'information, pas de détruire.
%
%Pousser le code sur un GitHub car les données sont anonymes par constructions.

\section{Mercredi 23 avril 2014}

Pour gérer le multiutilisateur sur une machine, on remplace le dossier repository par un dossier dont le nom est le début du UUID.

Le nom de la branche associé à un UUID est son hashé.

Un utilisateur pour continuer sa session sur une nouvelle machine (ou pas nouvelle) en saisissant son UUID.

Ceci permet d'éviter de continuer une partie anonyme d'un autre (en récupérant son UUID à partir d'un git clone).

Continuer session avec JList des sessions locales
Crée nouvelle session
Ajouter une session à partir d'un UUID
\end{document}