La Programmer's Learning Machine\cite{PLMQuinson} est une plate-forme d'enseignement de la programmation développée par Gérald Oster et Martin Quinson dans le cadre des cours à Telecom Nancy depuis 2007. Depuis sa création, l'outil a été utilisé dans le module de remise a niveau en informatique proposé aux élèves venant des CPGE en début de première année ainsi que dans le module \og techniques et outils pour programmer\fg{}. Cette plate-forme présente plusieurs avantages pédagogiques. Tout d'abord, elle permet la mise en œuvre pratique des notions abordées en cours de façon théorique et parfois déroutante pour les débutants. Ensuite, la représentation graphique des problèmes semble plus motivante pour les étudiants. De plus, l'exécution interactive et son cycle de développement/tests court permet aux étudiants d'évaluer eux-mêmes leur travail et de résoudre les problèmes de façon incrémentale. Enfin, la base relativement importantes d'exercices permet à chacun de s'entraîner individuellement à son rythme jusqu'à maîtriser les notions abordées. Dans son état actuel, l'environnement est surtout centré sur l'élève : il reste difficile pour l'enseignant de suivre les avancées des élèves pendant la séance afin de détecter les élèves ayant besoin d'aide, c'est-à-dire ceux qui sont bloqués sur un exercice.

C'est dans ce contexte que se situe notre PIDR (Projet Interdisciplinaire ou Découverte de la Recherche). Ce projet encadré par un enseignant-chercheur ou un chercheur permet un premier contact concret avec la recherche menée dans les laboratoires de l'Université, en particulier le LORIA, le CRAN et l'IECN. Dans le cas interdisciplinaire, le projet ne se déroule pas nécessairement en laboratoire et concerne des aspects liés à différentes disciplines autour de l'informatique. Dans notre cas, il s'agit des aspects pédagogiques liés à l'apprentissage de la programmation. Durant les quatre mois pendant lesquels se déroule le PIDR, nous devrons donc trouver et mettre en œuvre des solutions pour qu'il soit plus facile pour les enseignants comme pour les élèves de suivre les progrès réalisés lors de la résolution des exercices.
